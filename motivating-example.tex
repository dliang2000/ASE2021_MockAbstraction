\section{Motivating Example}
\label{sec:motivating-example}

In this section, we illustrate how our \textsc{MockDetector} tool finds a mock object created within a unit test case. Our tool identifies variables which have been assigned an object flowing from a mock creation site either through a forward flow may analysis (Soot-based analysis) or through specified declarative constraints (Doop-based analysis).

% explain focal methods first.

To motivate our work, consider Listing~\ref{lis:mockCall}, which presents a unit test case from the Maven project. Line 8 calls \textit{getRequest()}, invoking it on the mock object \texttt{session}. Line 12 then calls \textit{getToolchainsForType()}---this is the actual focal method whose behaviour is being tested. At the bytecode level, the two method invocations are indistinguishable with respect to mockness; to our knowledge, current static analysis tools cannot easily tell the difference between the method invocation on a mocked object on line 8 and the method invocation on a real object on line 12. This uncertainty would confound, for instance, a naive static analysis that attempts to identify focal methods.

While we were designing \textsc{MockDetector}, we observed several cases where mocked objects are stored in arrays and collections. Listing~\ref{lis:container} presents the before method \textit{setUp()} in class \texttt{NodeListIteratorTest.java} from commons-collections-4.4, where line 9 puts the mock \textsc{Node} objects in the private field \textsc{Node} array, which is later used in test cases. To find arrays containing mock objects, our analysis would first locate assignment statements containing ArrayRef, and would gather all the Locals or FieldRefs in these statements. The tool then checks whether any of these Locals or FieldRefs have already been marked as a mayMock in the analysis, and consequently the tool would mark the Local or FieldRef representing the array as an arrayMock, meaning the mockiness has propagated to this array container. 

\lstset{language=java,
	keywordstyle=\color{blue}\bfseries,
	commentstyle=\color{green},
	stringstyle=\ttfamily\color{red!50!brown},
        showstringspaces=false}
\lstset{literate=%
	*{0}{{{\color{red!20!violet}0}}}1
	{1}{{{\color{red!20!violet}1}}}1
	{2}{{{\color{red!20!violet}2}}}1
	{3}{{{\color{red!20!violet}3}}}1
	{4}{{{\color{red!20!violet}4}}}1
	{5}{{{\color{red!20!violet}5}}}1
	{6}{{{\color{red!20!violet}6}}}1
	{7}{{{\color{red!20!violet}7}}}1
	{8}{{{\color{red!20!violet}8}}}1
	{9}{{{\color{red!20!violet}9}}}1
}

\begin{lstlisting}[basicstyle=\ttfamily, caption={This code snippet illustrates an example from maven-core, where both the focal method and a method invocation on a mocked object occur in test \textit{testMisconfiguredToolchain()}},
basicstyle=\scriptsize\ttfamily,language = Java, framesep=4.5mm,
framexleftmargin=1.0mm, captionpos=b, xleftmargin=3.5ex, label=lis:mockCall]

@Test
public void testMisconfiguredToolchain() throws Exception {
	MavenSession session = mock( MavenSession.class );
	MavenExecutionRequest req = 
		new DefaultMavenExecutionRequest();
	when( session.getRequest() ).thenReturn( req );
	
	ToolchainPrivate[] basics =
	 	toolchainManager.getToolchainsForType("basic", session);
	
	assertEquals( 0, basics.length );
}

\end{lstlisting}

\begin{lstlisting}[basicstyle=\ttfamily, caption={This example illustrates a field array container holding mock objects from \textit{setup()} in \texttt{NodeListIteratorTest.java}.},
basicstyle=\scriptsize\ttfamily,language = Java, framesep=4.5mm, framexleftmargin=1.0mm, captionpos=b, xleftmargin=3.5ex, label=lis:container]

// Node array to be filled with mocked Node instances
private Node[] nodes;

@Test
protected void setUp() throws Exception {
	...
	
	// create mocked Node Instances and 
	// fill Node[] to be used by test cases
	final Node node1 = createMock(Element.class);
	final Node node2 = createMock(Element.class);
	final Node node3 = createMock(Text.class);
	final Node node4 = createMock(Element.class);
	nodes = new Node[] {node1, node2, node3, node4};
	...
}

\end{lstlisting}
