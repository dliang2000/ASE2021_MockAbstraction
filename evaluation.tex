\section{Evaluation}
\label{sec:evaluation}

\begin{table*}
	\centering
	\caption{LOC and Runtime information for each benchmark.}
	%	\begin{adjustbox}{width=0.1\textwidth}
	\begin{tabular}{lrrrrrr}
		\toprule
		Benchmark & Total LOC & Test Suite LOC & \thead{Soot Intra-proc \\ Total Time (s)} & \thead{Doop Intra-proc \\ Total Time (s)} & \thead{Soot Intra-proc \\ Mock Analysis (s)}  & \thead{Doop Intra-proc \\ Mock Analysis (s)} \\
		\midrule
		bootique-2.0.B1-bootique           		&  15530   & 8595   &  &  &  0.293   & 86    \\
		commons-collections4-4.4           		&  65273   & 36318  &  &  &  0.345   & 53        \\
		flink-core-1.13.0-rc1           		&  117310  & 49730  &  &  &  0.358   & 194        \\
		jsonschema2pojo-core-1.1.1         		&  8233    & 2885   &  &  &  0.282   & 180       \\
		maven-core-3.8.1   		           		&  38866   & 11104  &  &  &  0.192   & 126        \\
		micro-benchmark         		  		&  954     & 883	&  &  &  0.126   & 24        \\
		mybatis-3.5.6         		  			&  68268   & 46334  &  &  &  0.524   & 344        \\
		quartz-core-2.3.1        	  			&  35355   & 8423   &  &  &  0.215   & 40      \\
		vraptor-core-3.5.5         	  			&  34244   & 20133  &  &  &  0.340   & 251      \\
		\bottomrule
		Total         	  						&  384033  & 184405 &  &  &  2.675   & 1298      \\
	\end{tabular}
	%	\end{adjustbox}
	\label{tab:runtimes}
\end{table*}

\begin{table*}
	\centering
	\caption{Counts of Test/Before/After methods in test suite containing MayMock object, ArrayMock containers, and CollectionMock containers in Test/Before/After methods. (Intra-procedurally)}
	%	\begin{adjustbox}{width=0.1\textwidth}
	\begin{tabular}{lrrrr}
		\toprule
		Benchmark & \thead{\# of Test Related \\ Methods Invoked} & \thead{\# of Test Related \\ Methods with \\ MayMock (Intra)}  & \thead{\# of Test Related \\ Methods with \\ ArrayMock (Intra)} & \thead{\# of Test Related \\ Methods with \\ CollectionMock (Intra)} \\
		\midrule
		bootique-2.0.B1-bootique           		&  420        &  32  & 7 & 0       \\
		commons-collections4-4.4          		&  1152       &  3   & 1 & 1       \\
		flink-core-1.13.0-rc1           		&  1091       &  4   & 0 & 0       \\
		jsonschema2pojo-core-1.1.1           	&  145        &  76  & 1 & 0       \\
		maven-core-3.8.1	           			&  337        &  24  & 0 & 0       \\
		micro-benchmark         		  		&  59         &  43  & 7 & 25       \\
		mybatis-3.5.6         		  			&  1769       &  330 & 3 & 0       \\
		
		quartz-core-2.3.1         	  			&  218     	  &  7   & 0 & 0      \\
		vraptor-core-3.5.5         	  			&  1119       &  565 & 15 & 0      \\
		\bottomrule
		Total        	  						&  6310       &  1084  & 34 & 26    \\
	\end{tabular}
	%	\end{adjustbox}
	\label{tab:mocks}
\end{table*}

\begin{table*}
	\centering
	\caption{Comparison of Number of InstanceInvokeExprs on MayMock objects analyzed by Soot and Doop, and Total Number of InstanceInvokeExprs, in each benchmark's test suite.}
	%	\begin{adjustbox}{width=0.1\textwidth}
	\begin{tabular}{lrrr}
		\toprule
		Benchmark & \thead{Total Number \\ of Invocations} & \thead{\# of Invocations \\ on MayMocks (Soot)} & \thead{\# of Invocations \\ on MayMocks (Doop)} \\
		\midrule
		bootique-2.0.B1-bootique           		&  3366     &  99  & 99  \\
		commons-collections4-4.4       			&  12753    &  11  & 3  \\
		flink-core-1.13.0-rc1           		&  11923    &  40  & 40  \\
		jsonschema2pojo-core-1.1.1           	&  1896     &  276  & 282  \\
		maven-core-3.8.1           				&  4072     &  23   & 23  \\
		microbenchmark         		  			&  471      &  108  & 123      \\
		mybatis-3.5.6         		  			&  19232    &  575  & 577       \\
		quartz-core-2.3.1       	  			&  3436     &  21   & 21    \\
		vraptor-core-3.5.51        	  			&  5868     &  942  & 962    \\
		\bottomrule
		Total        	  						&  63017    & 2095  & 2125    \\
	\end{tabular}
	%	\end{adjustbox}
	\label{tab:invokes}
\end{table*}

\begin{table*}
	\centering
	\caption{Doop's runtime analysis with basic-only and context-insensitive options.}
	%	\begin{adjustbox}{width=0.1\textwidth}
	\begin{tabular}{lrrrrrr}
		\toprule
		Benchmark & \thead{Basic-only, \\ no-mock (s)} & \thead{Basic-only, \\ interproc-mock (s)} & \thead{Basic-only, \\ delta (s)} & \thead{Context-insensitive, \\ no-mock (s)}  & \thead{Context-insensitive, \\ interproc-mock (s)} & \thead{Context-insensitive, \\ delta (s)} \\
		\midrule
		bootique-2.0.B1-bootique           		&    &    &   &   &    &     \\
		commons-collections4-4.4           		&    &    &   &   &    &         \\
		flink-core-1.13.0-rc1           		&    &    &   &   &    &         \\
		jsonschema2pojo-core-1.1.1         		&    &    &   &   &    &        \\
		maven-core-3.8.1   		           		&    &    &   &   &    &         \\
		micro-benchmark         		  		&    & 	  &   &   &    &         \\
		mybatis-3.5.6         		  			&    &    &   &   &    &         \\
		quartz-core-2.3.1        	  			&    &    &   &   &    &       \\
		vraptor-core-3.5.5         	  			&    &    &   &   &    &      \\
		\bottomrule
		Total         	  						&    &    &   &   &    &       \\
	\end{tabular}
	%	\end{adjustbox}
	\label{tab:doop-runtimes}
\end{table*}

\begin{table*}
	\centering
	\caption{Counts of mock invocations for Doop in basic-only and context-insensitive options, and for inter-procedural and intra-procedural .}
	%	\begin{adjustbox}{width=0.1\textwidth}
	\begin{tabular}{lrrrr}
		\toprule
		Benchmark & \thead{Basic-only, \\ intraproc} & \thead{Context-insensitive, \\ intraproc} & \thead{Basic-only, \\ interproc} & \thead{Context-insensitive, \\ interproc} \\
		\midrule
		bootique-2.0.B1-bootique           		&    &    &   &       \\
		commons-collections4-4.4           		&    &    &   &        \\
		flink-core-1.13.0-rc1           		&    &    &   &         \\
		jsonschema2pojo-core-1.1.1         		&    &    &   &          \\
		maven-core-3.8.1   		           		&    &    &   &           \\
		micro-benchmark         		  		&    & 	  &   &           \\
		mybatis-3.5.6         		  			&    &    &   &          \\
		quartz-core-2.3.1        	  			&    &    &   &         \\
		vraptor-core-3.5.5         	  			&    &    &   &         \\
		\bottomrule
		Total         	  						&    &    &   &         \\
	\end{tabular}
	%	\end{adjustbox}
	\label{tab:doop-mock-invokes}
\end{table*}

The goal of our study is to correctly identify and trace mock objects as well as the method invocations in the test suite. To this end, we conduct quantitative and qualitative research focusing on two research questions:

\begin{quote}
	\emph{RQ 1: Are the mocks correctly identified and traced for each test method?}
\end{quote}

\begin{quote}
	\emph{RQ 2: Would this be helpful for existing static analysis tools?}
\end{quote}

\subsection{Quantitative Analysis}
\label{subsec:effectiveness}

We have evaluated \textsc{MockDetector} on 8 open-source benchmarks, along with a micro-benchmark that we developed to test our tool. Table~\ref{tab:runtimes} presents the LOC and runtime information for each benchmark. The 9 benchmarks include over 383 kLOC, with 184 kLOC in the test suites measured by SLOCCount. The runtime on Soot implementation's intra-procedural mock analysis  is obtained by summing up the soot timer's counts over the three transformers, which are responsible for analyzing field mocks defined by annotation or in \texttt{<init>}, field mocks defined in before methods, and mocks created in test cases, respectively. Meanwhile, our Doop implementation's runtime on intra-procedural mock analysis is obtained by taking the difference of two Doop runs on the same benchmark, where one run is associated with extra logic analyzing intra-procedural mocks. This would correctly reflect the run-time spent on analyzing intra-procedural mock invocations.  %Add reference for SLOCCount.

There are more test cases holding inter-procedural mocks (i.e., the mock object is created in a helper method and passed into the test case) in commons-collections and micro-benchmark. The inter-procedural analysis is currently in development and will be discussed in Section~\ref{sec:discussion}.

The Procedure Summaries produced after the analysis has indicated that the tracing of "mockiness" of variables and containers is also correct through the whole program. 

\subsection{Application}
\label{subsec:static}

The accuracy results in tracing intra-procedural mock objects or containers have indicated that \textsc{MockDetector} has the potential to be applied as a helper for existing static analysis tools. By adding proper adjustment, it could pass the mock information to the static analysis, so that the generated call graph may appropriately omit the methods invoked on mock objects, thus increasing its accuracy.

By running evaluation (also inter-procedurally) on more benchmarks, our tool would have the potential to finding the scenario where developers prefer using mock objects for dependencies, and subsequently providing mock suggestions.
