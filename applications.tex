\section{A Survey of Applications}
\label{sec:applications}

Our work supports test-to-code traceability. In this section, we
sketch several applications of our work to this important problem,
following Ghafari et al~\cite{ghafari15:_autom} but specializing to
our mock analysis.

\paragraph{Test case comprehension} xUnit tests are snippets of arbitrary
code. In unpublished research, we have established that most tests are
simple straight-line code. However, this code by necessity interacts
with the system under test in potentially complicated ways. Hence,
understanding what a test case is doing can be difficult: previous
work has described experiments where developers are asked to complete
program understanding tasks on tests (e.g. identify-the-focal-method),
and finds that this is surprisingly difficult. Our mock analysis helps
segregate tests into parts that are mock-related and parts that are
not mock-related.  It is fairly obvious to a human reader that the
part of a test that is calling mock library methods such as
\texttt{thenReturns()} is setting up mock expectations, but we've seen cases
where this is less obvious.\todo{example?}

Knowing what is a mock can help developers debugging test case
failures get into the right mindset, in two ways: 1) when looking at
mock calls, they can conclude that these mock calls should not
directly be causing the test failures; but also, 2) if the mock calls
are now returning incorrect values (perhaps due to program evolution),
then it may be appropriate to update the recorded values.

\paragraph{Code recommendation}
To amplify the previous point, a key tenet of test-to-code
traceability is that when the code is updated, related tests may also
need to be updated. Integrated Development Environments can search for
all tests referring to a particular fragment of code. Mock analysis
can augment the information available to the developer (maintainer) by
giving them additional information about whether they are updating
tests that depend on the changed code as a mock or whether the tests
are in fact testing the changed code itself.

\paragraph{Automatic debugging and repair}
Relies on failing tests to id repair subjects

\paragraph{Automated refactoring/code generation}
We formulated the mock analysis problem as a side problem which needed
to be solved in the service of a deeper problem: automatically
generating useful tests. In that context, we needed to know which call
was to the focal method of a test---we wanted to create additional
tests based on those focal methods, but not based on mock objects.

\paragraph{Extracting API usage examples}
write stuff here

All of our applications show that the treatment of mocks and non-mocks
when editing test cases (either manually or automatically) should be different.
Furthermore, especially for automatic treatments of test cases, a static
analysis determining which invocations are mock invocations is key to effective
treatment of that case.

%* code recommendation for editing test cases
%* automated debugging and repair that rely on failing tests to id repair subjects
%* automated refactoring
%* extracting API usage examples [Ghafari ICPC 2014]
