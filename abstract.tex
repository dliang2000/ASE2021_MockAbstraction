\begin{abstract}
	Software dependencies are ubiquitous and may pose problems during testing, because creating usable objects from dependencies is often complicated.
	Developers, therefore, often introduce mock objects to stand in for dependencies during testing. However, to our knowledge, no static analysis framework provides a tool to automatically identify mock objects created in the unit test cases. The lack of mock object detection can decrease the precision of static analyses, as they are unable to separate methods invoked on mock objects from methods invoked on actual objects. 
	
	In this paper, we introduce MockDetector, a technique to identify mock objects. It is able to detect common Java mock libraries' APIs that create mock objects, checking whether there is a call to a mock creation site, followed by a forward flow must analysis, which aims to include all locals that are indeed mock objects on all possible paths, as well as the containers such as array or collection that have must mock objects stored. Implications of understanding which objects are mock objects include helping static analysis tools identify which dependencies' methods are actually tested, versus mock methods being called.

\end{abstract}
